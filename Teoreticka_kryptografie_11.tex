\documentclass[a4paper,12pt,titlepage]{article}
\usepackage[IL2]{fontenc}
\usepackage[utf8]{inputenc}
\usepackage{a4wide}
\usepackage[czech]{babel}
\usepackage{amsfonts, amsmath, amsthm, amssymb}
\usepackage[left=1.5cm,right=1.5cm,top=1.5cm,bottom=1.5cm]{geometry}
\usepackage{mdwlist}
\usepackage{textcomp}

\def\nadpis#1{{\bigskip\large\bf\noindent{#1}}\par\bigskip}
\def\podnadpis#1{{\bigskip\bf\noindent#1\medskip\par}}
\def\definice{\noindent {\bf Definice: }}
\def\priklad{\noindent {\bf Příklad: }}
\def\tvrzeni{\noindent {\bf Tvrzení: }}
\def\veta{\noindent {\bf Věta: }}
\def\dukaz{\noindent {\bf Důkaz: }}
\def\qed{{\hfill{$\square$}}}
\def\m#1{{\mathcal{#1}}}
\def\sskip{\medskip}

\def\cm{{\rm cm}}
\def\Comm{{\rm Comm}}
\def\ramecek#1{\hbox{\vrule\vbox{\hrule\vskip 2pt
#1
\hrule}\vrule}}

\begin{document}



\podnadpis{4. 1. 2018}

\podnadpis{Zero-knowledge pro NP}

Pokud existují OWP (jednosměrné permutace), pak každý jazyk v NP má ZK (zero-knowledge) interaktivní důkaz.

Konstrukce pro jeden NP-úplný jazyk je postačující

\podnadpis{Idea Protocol} pro $L_{3\rm COL}$ (obarvení grafu třemi barvami)

Společný vstup: graf $G$

Proverův vstup: validní obarvení $G$ třemi barvami

1) Prover zvolí permutaci $\pi\leftarrow S_3$ barev, obarví $G$ pomocí toho obarvení (po permutaci), zakryje graf pomocí kalíšků

2) Verifier vybere hranu grafu $e\in E$ a pošle ji proverovi

3) Ten odkryje kalíšky na hraně

4) Verifier akceptuje, pokud je $e$ obarvena různými barvami
\vskip 5pt

1)--4) opakujeme $|V||E|$krát

\vskip 10pt
{\noindent\bf Důkaz správnosti:}
\vskip 5pt

{\bf Soundness} Pokud graf není obarvitelný třemi barvami, verifier zvolí špatně obarvenou hranu s pravděpodobností aspoň $1\over |E|$

Pravděpodobnost podvodu při $|V||E|$ opakováních je  $(1-{1\over|E|})\approx e^{-|V|}=e^{-n}$.

{\bf Completeness} Verifier akceptuje s pravděpodobností $1$ pro $x\in L_{3\rm COL}$.
\vskip10pt

Co kdyby prover měl nějaké kalíšky, se kterými si může dělat, co chce?

Verifier vidí dvě náhodné různé barvy, je to zero-knowledge. Nepozná, jestli mu prover lže.

Kalíšky = commitment schemes

Prover musí být vždy schopen to otevřít jen na tu hodnotu, která v krabičce/kalíšku byla.

{\noindent \bf Protokol mezi Sender a Receiver} -- má dvě fáze

1) commit -- sender pošle receiverovi commitment c pro hodnotu v

2) reveal -- S pošle R decomitment r pro hodnotu v
\vskip 10pt

{\noindent \bf Obrázek:}

\ramecek{\halign{\hskip 1pt#\hskip 5pt&#\hskip5pt&#\cr
S&&R\cr
$\Comm_r(v),c$&$\rightarrow$&commit\cr
$v,r$&$\rightarrow$&reveal $c=\Comm_r(v)$\cr
}}
\vskip 10pt

\definice(commitment scheme)

PPT algoritmus Comm nazýváme commitment scheme, pokud pro polynom $l$ platí:

1) {\bf Binding} $\forall n\in\mathbb{N},v_0,v_1\in\{0,1\}^n,r_0,r_1\in\{0,1\}^{l(n)}$ platí:

$$\Comm(v_0,r_0)\neq \Comm(v_1,r_1)$$

tzv. Perfectly-binding

2) {\bf Hiding} $\forall PPT$ distinguishera D existuje negligible $\varepsilon$ tž. $\forall n\in N,r_0,r_1\in\{0,1\}^n$

$$|{\rm Pr}_{r\in\{0,1\}^{l(n)}}[D[\Comm(r_0,r))=1]-{\rm Pr}[D(\Comm(r_1,r))=1]|\leq\varepsilon$$ přes náhodné mince D.
\vskip 10 pt

{\noindent \bf Tvrzení} pokud existují OWP, pak existují schémata pro commitment.

\dukaz konstrukce pro commitment pro jeden bit

(pomocí hybridního argumentu lze rozšířit pro libovolně mnoho bitů)

Nechť f je OWPs hardcore bitem H. Pak můžeme definovat commitment $(b,r)$ jako $f(r)||b\oplus h(r)$

{\bf Binding:} Z konstrukce na základě toho, že f je permutace

{\bf Hiding:} stejný důkaz jako v konstrukci PRG z OWP

\vskip 5pt

%commitment je velmi užitečná věc, najen v životě, atk i v kryptografii

Commitment je důležitý stavební prvek kryptografických protokolů

\podnadpis{Protokol pro $L_{3COL}$}

Společný vstup: $G=(V,E0)$ $|V|=n$

Proverův vstup: wittness $y(c_1,\dots c_n), c_i\in \{1,2,3\}, c_i'=\pi(c_i)$

1) Prover zvolí $\pi\leftarrow S_3$. Pro $i\in\{1,\dots,n\}$ pošle verifierovi $\Comm_{r_i}(c_i')$

2) Verifier zvolí $(i,j)\in E$ a pošle $(i,j)$ P

3) P pošle V $(c_i',r_i)$ a $(c_j',r_j)$

4) V akceptuje, pokud $c_i'\neq c_j'$
\vskip 5pt

1)--4) opakujeme $n|E|$krát
\vskip 10pt

\tvrzeni protokol je zero-knowledge interaktivní důkaz pro $|_{COL}$, pokud comm je commitment scheme.

\dukaz completeness + soundness stejné (díky binding)
\vskip 10pt

{\noindent \bf Obrázek:}

\ramecek{\halign{\hskip 1pt#\hskip 5pt&#\hskip 5pt&#\cr
P&&V\cr
$\cm{}_1,\dots \cm{}_n$&$\rightarrow$&commit\cr
&$\leftarrow$&$(i,j)$\cr
$(c_i',r_i),(c_j',r_j)$&$\rightarrow$&reveal $c=\Comm_r(v)$\cr
}}
\vskip 10 pt

\podnadpis{Simulátor:}

$S(G,Z)$:

1) vyber náhodnou hranu $(i_s, j_s)$ grafu a dvě náhodné barvy za podmínky, že jsou různé

definujme $c_k'=1$ pro $k\notin\{i,j\}$

2) zkonstruuj commitment $cm_i$ pro všechny $c_i'$

emuluj $V^*(x,z,cm_1\dots,cm_n)$

\noindent Nechť odpověď $V^*$ je $(i,j)$

3) Pokud $(i,j)=(i_s,j_s)$, pak otevři $cm_{i_s}$ a $cm_{j_s}$ a vrať odpovídající ${\rm view}_{V^*}=(X,Z, r^*,cm_1,\dots,cm_n, (c_{i_S}',r_{i_S}),(c_{j_S}',r_{j_S}))$

V opačném případě se vrať na 1)

4) Pokud neuspějeme ani po $n|E|$ iteracích, vrať {\tt fail}
\vskip 10pt


\podnadpis{Simulátor pouze pro jednu iteraci protokolu} (pro všechny iterace ze sekvenční kompozice $Z_k$)

Zbývá ukázat, že $\forall$PPT D existuje negl. $\varepsilon$ tž. $\forall x\in L_{3\rm COL},y\in R_{L_{3\rm COL}}(x),z\in\{0,1\}^*$:

$$|{\rm Pr}[D({\rm view\ }V^*(P(x,y)\leftrightarrow V^*(x,z)))=1]-{\rm Pr}[D(S(x,z))=1]|\leq\varepsilon(n)$$

Pro spor předpokládejme, že existuje PPT D, který rozliší $\{{\rm view\ }(P(x,y)\leftrightarrow V^*(x,z))\}$ a $S(x,z)$ s pravděpodobností $\geq{1\over p(n)}$ pro polynom $p$ a nekonečně mnoho $x\in L_{3\rm COL}, y\in R_{3\rm COL}(x)$ a $z\in \{0,1\}^*$

\vskip 10pt
\podnadpis{Hybridní simulátory}

$S'(x,z,y)$: postupuje jako $S$ kromě volby $c'_{i_S}$ a $c'_{j_S}$

spočítá $\cm_{i_S}$ a $\cm_{j_s}$ jako $P[c_{i_S}'=\pi(c_{i_S})]$ a $P[c_{j_S}=\pi(c_{j_S})]$ pro $\pi \leftarrow S_3$

distribuce $\{S(x,z)\}$ a $\{S'(x,z,y)\}$ jsou identické

$S''(x,z,y)$: postupuje jako $S$, commitment pro obarvení kompletně jako prover

pokud se $V^*$ zeptá na $(i,j)$ různé od $(i_S,j_S)$, pak iteruje znovu a případně vrátí {\tt fail}

pokud $S''$ nevrátí {\tt fail}, pak jsou distribuce
$\{viewV^*(P(x,y)\leftrightarrow V^*(x,z))\}$ a $\{S''(x,zy)\}$ identické

jsou-li $(i,j)$ a $(i_S,j_S)$ nezávislé, pak $S''$ vrátí {\tt fail} s pravděpodobností$\leq (1-{1\over |E|})^{n|E|}\approx e^{-n}\leq{1\over 2p(n)}$

D rozliší tyhle distribuce s pravděpodobností $\leq{1\over 2p(n)}$

pro distribuce S' a S'' platí, že D je rozlišuje s ppstí alespoň ${1\over 2p(n)}$

$S\equiv S'$ (ppst pro rozlišení $\geq{1\over 2p(n)}$ $S''$ (pravděpodobnost pro rozlišení $\leq {1\over 2p(n)} (P,V^*)$ -- schematicky to, co je napsáno výše

Pro alespoň jeden z hybridů budeme mít přechod, který nám umožní prolomit hiding toho commitmentu

Shrnutí: mezi $S'$ a $S''$ lze zkonstruovat polynomiálně ($q(n)$) mnoho hybridů, které se liší v jednom commitmentu , z toho musí existovat dvojice hybridů, které lze rozlišit s ppstí alespoň $1\over q(n)p(n)$, tedy spor s hiding property commitmentu \hfill$\square$

\podnadpis{Aplikace}

{\bf ZeroCash} -- anonymní varianta BTC na základě ZK

${\rm coin}=(r,sn,\rm cm)$, kde $\cm=\Comm_r(\rm sn)$ (serial number)
\vskip 10pt

\hbox{\hskip\parindent\vbox{\halign{#\cr
na ledger umístíme transakci $\rm TX_{Mint}$\cr
\noalign{\hrule\vskip 2pt}
cm; 1BTC$\rightarrow$POOL-BEC\hfil\cr}}}
\vskip 10pt

pro utracení zveřejníme sn a dokážeme pomocí ZK

znám $r$ tž. $\cm=\Comm_r(\rm sn)$ je v CMList
\vskip 10pt

\hbox{\hskip\parindent\vbox{\halign{#\cr
$\rm tx_{spent}$\cr
\noalign{\hrule\vskip 2pt}
sn,$\pi$,1ZEC=1BTC\cr}}}
\vskip 10pt

máme tzv. ZK-SNARK $=$  ZK-sufficient non-interactive argument of knowledge

$\pi$ -- důkaz má konstantní délku a je neinteraktivní

--veřejně ověřit!

pro aplikaci výše můžeme dokazovat, že je Listem pro CMTree s kořenem Root

2014 -- ZK-verifikace transakce u BTC
\vskip 10pt

\podnadpis{Předpoklady} \noindent dělíme na :

{\bf Falsifiable} -- lze falsifikovat (neexistuje algoritmus na\dots)

{\bf Non-falsifiable} -- \uv{pro každý vstup existuje algoritmus\dots}

{\bf False} -- můžeme pro tento předpoklad dokázat, co chceme :)

\end{document}