\documentclass[a4paper,12pt,titlepage]{article}
\usepackage[IL2]{fontenc}
\usepackage[utf8]{inputenc}
\usepackage{a4wide}
\usepackage[czech]{babel}
\usepackage{amsfonts, amsmath, amsthm, amssymb}
\usepackage[left=1.5cm,right=1.5cm,top=1.5cm,bottom=1.5cm]{geometry}
\usepackage{mdwlist}
\usepackage{textcomp}

\def\nadpis#1{{\bigskip\large\bf\noindent{#1}}\par\bigskip}
\def\podnadpis#1{{\bigskip\bf\noindent#1\medskip\par}}
\def\definice{\noindent {\bf Definice: }}
\def\priklad{\noindent {\bf Příklad: }}
\def\tvrzeni{\noindent {\bf Tvrzení: }}
\def\veta{\noindent {\bf Věta: }}
\def\dukaz{\noindent {\bf Důkaz: }}
\def\qed{{\hfill{$\square$}}}
\def\m#1{{\mathcal{#1}}}
\def\sskip{\medskip}
\renewcommand\epsilon\varepsilon

\begin{document}

\podnadpis{12. 10. 2017}

\tvrzeni Pokud $(G,E,D)$ splňuje perfect secrecy, pak platí $|\m K| \geq |\m M|$.

\dukaz Nechť $|\m K| < |\m M|$. Nechť $M$ je rovnoměrné pravděpodobnostní rozdělení na $\m
M$. Nechť $c \in \m C$:

$Pr[E_k(M) = c] > 0$

$\m M(c) := \{ m \mid m = D_k(c) \mbox{ pro $k \in \m K$}\}$

Platí $|\m M(c)| \leq |\m K| < |\m M|$, tedy $\exists m' \in \m M: m' \notin \m M(c)$.

Závěrem dostáváme: $Pr[M = m' \mid \m C = c] = 0 \neq Pr[M= m']=1/|\m M|$  \qed


Kde můžeme slevit v nárocích na bezpečnost? První možností je předpokládat, že Eve je
stále výpočetně neomezená, což vede na statistical security.

\podnadpis{Statistical security}

\definice Nechť $X, Y$ jsou náhodné proměnné nad $S$. Řekneme, že $X, Y$ jsou
\textit{statistically $\epsilon$-in\-dis\-tin\-gui\-shable}, pokud $$\forall T 
\subseteq S: |Pr[X \in T] - Pr[Y \in T]| \leq \epsilon.$$
$T$ je statistický test.

\definice $(G,E,D)$ splňuje \textit{statistical $\epsilon$-indistinguishability}, pokud
$\forall m_0 \, \forall m_1 \in \m M$ jsou náhodné proměnné $E_k(m_0)$ a $E_k(m_1)$
statistically $\epsilon$-indistinguishable, tj. $$\left|Pr[E_k(m_0) \in T] - Pr[E_k(m_1) \in
T]\right| \leq \epsilon.$$

Adversary má pravděpodobnost $\leq \epsilon$ nalézt $m$ z $c$.

Podobně jako pro perfect secrecy platí $|\m K| \geq (1-\epsilon)|\m M|$.

\podnadpis{Computational security}

Jaká je výhoda kryptografie s důkazy? Z důkazů lze odvodit rozumný parametr, který nám dá
délky klíčů, a napoví, s jakou pravděpodobností adversary prolomí protokol.

\podnadpis{Asymptotická formalizace}
\begin{itemize}
\item "security parameter", Alice a Bob zvolí $n \in \mathbb{N}$
\item Efektivní adversary PPT (probabilistic polynomial time), pro každý security parametr
má program běžící čas $poly(n)$ (neuniformní)
\item $(G,E,D)$ v fixním polynomiálním čase
\item $|\m M|$ závisí na $n$: $\m M = \bigcup_n \m M_n$, kde např. $\m M_n = \{0,1\}^n$
\end{itemize}

\definice Funkce $\epsilon: \mathbb{N} \rightarrow [0,1]$ je \textit{negligible
(zanedbatelná)}, pokud: $$\forall c \in \mathbb{N} \,\exists n_c \in \mathbb{N}\, \forall n >
n_c: \epsilon(n) < \frac{1}{n^c}$$

Příkladem negligible funkcí jsou $2^{-n}$, $n^{-\log(n)}$, $2^{-\sqrt{n}}$.

\definice $(G,E,D)$ na prostoru zpráv $\m M = \bigcup_n \m M_n$, kde délka všech zpráv v
$\m M_n$ je stejná, splňuje \textit{indistinguishability ciphertextů}, pokud $\forall$ PPT
$A$ $\exists$ negligible $\epsilon$ takové, že
$$\forall m_0, m_1 \in \m M_n: |Pr[A(E_k(m_0)) = 1] - Pr[A(E_k(m_1)) = 1]| \leq
\epsilon(n)$$
Pravděpodobnost je přes $k \leftarrow G(1^n)$ a náhodné mince $E$ a $A$. Ciphertext má
vždy délku $\geq n$.

Definice, kterou nebudeme používat: $(\epsilon,t)$ secure, pokud $\forall A$ běžící v čase
$t$ platí podmínka.

Asymptotická vs. konkrétní definice: $t$ ve specifickém výpočetním modelu ($2^{100}$ cyklů
CPU) $(G,E,D)$ v čase $\ll t$, $\epsilon \leq 2^{-100}$

Cíl: $(G,E,D)$, kde $|\m K| \ll |\m M$.

Příklad, který nefunguje:

Pravděpodobnostní OTP:\\
$\m K = \{0, 1\}^n$ $\m M = \{0,1\}^{2n}$\\
$E_k(m) = i_1, \dots, i_{2n} \leftarrow \{1, \dots, n\}$\\
$c = (i_1, \dots, i_{2n}, m \oplus (k_{i_1}, k_{i_2}, \dots, k_{i_{2n}})$\\
$G(1^n): k \leftarrow \{0,1\}^n$

\definice $(G,E,D)$ nad $\m M = \bigcup_n \m M_n$, kde délka všech zpráv v
$\m M_n$ je stejná, splňuje \textit{guessing indistinguishability
ciphertextů}, pokud $\forall$ PPT $A$ $\exists$ negligible $\epsilon$ tak, že $A$ zvítězí
v následující hře s pravděpodobností nejvýše $\frac{1}{2} + \epsilon(n)$.

\begin{enumerate}
\item $A$ zvolí $m_0, m_1 \in \m M_n$
\item $k \leftarrow G(1^n)$ a $b \leftarrow \{0, 1\}$
\item $A$ dostane $E_k(m_b)$ a vrátí $b'$
\item $A$ zvítězí, pokud $b = b'$
\end{enumerate}

\tvrzeni $(G,E,D)$ splňuje indistinguishability ciphertextů právě tehdy, když splňuje
guessing indistinguishability ciphertextů. 

\dukaz "$\Leftarrow$" Mějme $\m A_i$ pro $m^*_0, m^*_1 \in \m M$, kde
$$|Pr[\m A_i(E_k(m_0^*)) = 1] - Pr[\m A_i(E_k(m_1^*)) = 1]| \geq \frac{1}{p(n)}$$
pro $p \in poly(n)$.

Zkonstruujeme $\m A_{gi}$: 1) zvol $m_0^*$, $m_1^*$ 3) pro c odpověz $\m A_i(c)$

$\m A_{gi}$ zvítězí s pravděpodobností $\geq \frac{1}{2} + \frac{1}{2p(n)}$.

"$\Rightarrow$" Obdobně.

\definice $(G,E,D)$ nad $\m M = \bigcup_n \m M_n$, kde délka všech zpráv v
$\m M_n$ je stejná, splňuje \textit{semantic security}, pokud $\forall$
PPT $A$ $\exists$ PPT $A'$ takový, že pro všechna rozdělení $M$ nad $\m M$ a každou funkci
$f \colon \m M \rightarrow \{0, 1\}^*$ platí:
$$Pr[A(E_k(M)) = f(M)] \leq Pr[A'(1^n) = f(M)] + \mbox{negl}(n)$$

Autory definice jsou Goldwasser a Micali. $f$ je libovolná funkce jako například $f(m) =
m$ nebo $f(m)=\mbox{50. bit $m$}$.  



\tvrzeni (Bez důkazu) $(G,E,D)$ splňuje semantic security právě tehdy, když splňuje
indistinguishability ciphertextů.

\end{document}
