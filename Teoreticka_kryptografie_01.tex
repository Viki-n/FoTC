\documentclass[a4paper,12pt,titlepage]{article}
\usepackage[IL2]{fontenc}
\usepackage[utf8]{inputenc}
\usepackage{a4wide}
\usepackage[czech]{babel}
\usepackage{amsfonts, amsmath, amsthm, amssymb}
\usepackage[left=1.5cm,right=1.5cm,top=1.5cm,bottom=1.5cm]{geometry}
\usepackage{mdwlist}
\usepackage{textcomp}

\def\nadpis#1{{\bigskip\large\bf\noindent{#1}}\par\bigskip}
\def\podnadpis#1{{\bigskip\bf\noindent#1\medskip\par}}
\def\definice{\noindent {\bf Definice: }}
\def\priklad{\noindent {\bf Příklad: }}
\def\tvrzeni{\noindent {\bf Tvrzení: }}
\def\veta{\noindent {\bf Věta: }}
\def\dukaz{\noindent {\bf Důkaz: }}
\def\qed{{\hfill{$\square$}}}
\def\m#1{{\mathcal{#1}}}
\def\sskip{\medskip}
\begin{document}
\podnadpis{Podmínky splnění předmětu}

\noindent Bude celkem 6 úkolů, cvičení bude virtuální pouze prostřednictvím
zadávání a řešení úkolů (zadávány na konci přednášky). Každý by měl odevzdat
všechny, je nutné získat alespoň se 60 \% bodů.

Na stránce předmětu je literatura, kvalitní je Katz + Lindell -- Introduction
to Modern Crypto a Goldreich -- Foundation of Cryptography vol 1.
 
\noindent Ani jedna z nich není dostupná v elektronické podobě. Proto aby
existoval použitelný učební materiál, budeme my dělat zápisy, ty budou
korigovány a zveřejňovány. Za zápis z přednášky není potřeba odevzdat 1
úkol. Zapsané zápisy posílat na adresu {\sl hubacek $^\star$at$^\star$ iuuk.mff.cuni.cz}.

Hlavní část předmětu by mělo být plnění úkolů, deadline bude 2 týdny, na zkoušce
si popovídáme o nějakých problémech.

\nadpis{Foundations of Theoretical Cryptography}

\podnadpis{Historie}

\noindent První zmínky o kryptografii z r. 500 př. n. l., ale až do 2. sv.
války byla kryptografie \uv{tajemným uměním}, až za 2. sv války začala být
vědou.

Vždy nejprve někdo navrhl systém, někdo jiný na něj navrhl útok, někdo ho
opravil a potom se cyklil 2. a 3. bod.

Systém byl považován za bezpečný, pokud ho toho času nikdo neuměl prolomit.

\begin{itemize}
\item 1940 -- Shannon přišel s rigorózním přístupem ke kryptografii
\begin{itemize}
\item definice perfect secrecy.

\item ukázal, že perfect secrecy je nepraktická.
\end{itemize}

\item 1970 -- Diffie a Hellman
\begin{itemize}
\item definice Computational security

\item přišli na to, že nepotřebujeme systémy, které nelze prolomit, stačí, když
je nelze prolomit efektivně $\Rightarrow$ kryptografii lze budovat na obtížných
problémech.

\item win-win situace -- pokud se podaří prolomit šifru, tak máme alespoň
řešení na nějaký obtížný problém.

\end{itemize}

\item1980 -- Goldmasser a Micali
\begin{itemize}

\item\uv{Rozumné} definice pro kryptografii

\item konstrukce pro dané definice na základě předpokladů jako např. $P \neq
NP$, faktorizace je obtížná\dots

\item Pro to, abychom mohli věřit, že kryptografie existuje, je potřeba věřit,
že $P \neq NP$ a většinou je potřeba i nějaký silnější předpoklad.

\end{itemize}
\end{itemize}

\podnadpis {Šifrování:}

\noindent Máme Alici a Boba, kteří by spolu rádi komunikovali, ovšem mají pouze
kanál, který odposlouchává Eva (Eavesdropper)

\begin{itemize}

\item(realistická situace, postupně existovali kurýři, pošta, email, mobil\dots) 

\item\uv{encryption scheme}/secret code/secret key

\item dříve fungovala Stegamografie -- způsob, jak schovat data do jiné
legitimně vypadající zprávy, v dnešní době ale tento přístup již není zajímavý.

\end{itemize}

\podnadpis{Public-key cryptography}

\begin{itemize}

\item bez sdíleného klíče

\item autentizace a integrita dat

\begin{itemize}

\item Alice ověří zprávu od Boba (může si být jistá, že zpráva je opravdu od
něj a nebyla modifikována)

\end{itemize}

\item Mimo komunikaci: secure computation

\begin{itemize}

\item protokoly pro aukce, e-cash, voting, kdy máme mnoho účastníků, adversary
je interní -- chceme docílit toho, aby nikdo např. nemohl ovlivnit volby mimo
váhu vlastního hlasu.

\end{itemize}

\end{itemize}

\podnadpis{Vrstvy kryptografie}

\begin{enumerate}

\item Výpočetně složité problémy

\begin{itemize}

\item základní stavební kameny

\item vycházejí z teorie složitosti nebo z výpočetní teorie čísel (v některých grupách diskrétní logaritmus, faktorizace,\dots)

\item nutné předpoklady:

\item u Shannona -- šifry použitelné, i když má Eva neomezenou výpočetní sílu. Zde nám ale stačí předpoklad, že náš problém je obtížný (nevíme ale, jestli nějaký takový opravdu existuje)

\end{itemize}

\item Kryptografická primitiva

\begin{itemize}

\item základní \uv{kryptografický úkol} -- šifrování, digitální podpis,
pseudonáhodné generátory, zero-knowledge proof

\item konstrukce s důkazy bezpečnosti

\end{itemize}

\item Protokoly

\begin{itemize}

\item aukce, voting,

\item též máme důkazy bezpečnosti, ale většinou se pouze odkazují na důkazy
předchozí vrstvy

\end{itemize}

\item Systémy

\begin{itemize}

\item TLS/SSL -- důkazy bezpečnosti většinou ještě neexistují, též jsou ale
typicky nutné dobré stavební prvky z předchozích vrstev.

\end{itemize}

\end{enumerate}

\podnadpis{Private-key Encryption (Symetric)}

\noindent Alice a Bob mají sdílený klíč
Předpoklady:

\begin{itemize}

\item Kerchhoffův princip: Předpokládáme, že encryption i decription algoritmy
jsou známé, adversary nezná klíč. (dobrý předpoklad, algoritmy vždycky nakonec
uniknou)

\item insecure channel -- adversary vidí všechny správy, ale je pasivní (nemůže
měnit, ničit, či tvořit vlastní zprávy)

\end{itemize}

\penalty -1000

\definice {\it Private-key encryption scheme} s prostorem klíčů $\m K$,
prostorem zpráv $\m M$ a prostorem ciphertextů $\m C$ je trojice algoritmů
$(G,E,D)$, kde:

\begin{itemize}

\item $G$ generuje klíče z $\m K$, značíme $k \leftarrow G$, $k \in \m K$

\item $E$ pro klíč $k \in \m K$ a zprávu $m \in \m M$ vrací ciphertext $c \in
\m C$, značíme $c\leftarrow E_k(m)$.

\item $D$ pro klíč $k \in \m K$ a cipher text $c \in \m C$ vrací zprávu $m$,
značíme $m=D_k(c)$.

\end{itemize}

Šipky v definicích znamenají, že $G$ a $E$ není deterministický ale
pravděpodobnostní, $D$ je naopak deterministický (to nás neomezuje, tyto
případy se na sebe dají převést)

Požadujeme korektnost, tedy $\forall k \in \m K, m \in \m M, D_k(E_k(m))=m$.

Typicky $|\m M| \neq |\m C|$.

Co od algoritmu chceme v rámci bezpečnosti
- požadovat, aby ciphertext nic neříkal o klíči nefunguje.

\definice Private-key encryption scheme splňuje {\it perfect
indistinguishability}, pokud $\forall m_1, m_2 \in \m M, c \in \m C:
Pr_{k\leftarrow \m K}[E_k(m_1)=c]=Pr_{k\leftarrow \m K}[E_k(m_2)=c]$,
pravděpodobnost počítáme přes pravděpodobnost $k\leftarrow G$ a náhodné mince v
$E$. \sskip

Pokud jsme schopní splnit tohle, znamená to, že adversary z ciphertextu opravdu
nezjistí vůbec nic, dokonce i kdyby znal část zprávy (klidně celou až na jeden
bit).

\priklad
Shift cipher (Caesarova šifra)
$\m K=\{0,1,\dots,25\}$,$\m M=\m C=\m K ^\ell$
Substitution cipher
$K=\Phi(25)$ (=permutace na $\{0,1,\dots,25\}$), $\m M=\m C=\m K^\ell$
One-time pad(Vernamova šifra)
$\m K=\m C=\m M=\{0,1\}^\ell$
$E_k(m)=m \oplus k$, $D_k(m) = m \oplus k$\sskip

\tvrzeni Caesarova šifra pro zprávy délky $\ell \geq 2$ nesplňuje perfect ind.\sskip

\dukaz $m_1=aa$, $m_2=ab$, $c=xy$,
$Pr_{k\leftarrow \m K}[E_k(m_2)=c]=\frac 1{26}$
$Pr_{k\leftarrow \m K}[E_k(m_1)=c]=0$ \qed\sskip

\tvrzeni OTP splňuje perfect ind.\sskip

\dukaz
$Pr_{k\leftarrow \m K}[E_k(m)=c] = Pr_{k\leftarrow \m K}[n=m \oplus c |
k\leftarrow \{0,1\}^\ell]=2^{-\ell}$ \qed \sskip

\definice $M$ je pravděpodobnostní rozdělení na $\m M$. Pak PK encryption
scheme splňuje {\it Shannon secrecy} vzhledem k $M$, pokud $\forall m \in \m M,
c \in \m C$ taková, že $Pr[E_k(m)=c]>0$ platí $Pr[M=m | E_k(m)=c]=Pr[M=m]$

Pravděpodobnost přes $M$, $k \rightarrow \m K$ a $E$. \sskip

\tvrzeni PK-encryption scheme splňuje perf. ind. právě když splňuje Shannon
secrecy (vzhledem ke každému $M$, $Pr[M=m]>0 \forall m \in M$) Pak říkáme, že
splňuje perfect secrecy.\sskip

\dukaz "$\Rightarrow$"

Bayesova věta: Pokud platí $Pr[B]>0$, pak $\forall A:$ $$Pr[A|B]={Pr[B|A]Pr[A]
\over Pr[B]}.$$

Z Bayesovy věty platí $$Pr[M=m|E_k(m)=c] =
\frac{Pr[E_k(m)=c|M=m]Pr[M=m]}{Pr[E_k(m)=c]}.$$ Z perf. ind. ale platí, že
$Pr[E_k(m)=c|M=m]=Pr[E_k(m)=c]$, tyto členy můžeme proti sobě tedy pokrátit
a dostaneme $Pr[M=m|E_k(m)=c] = Pr[M=m]$, čímž je tato implikace dokázána.

\bigskip \noindent Zbytek důkazu na příští přednášce.

\end{document}
