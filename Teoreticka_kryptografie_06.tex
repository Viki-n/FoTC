\documentclass{article}
\usepackage{mathtools}
\usepackage{amsfonts}
\usepackage[utf8]{inputenc}

\begin{document}

%poznámka: k v pk a sk má být velké

\section{Public key encryption}

Umožnilo revoluci v kryptografii tím,
že odstranilo problém private key encryption
-- potřebu domluvit se na klíči předem.
To vycházelo z historického náhledu na šifrování,
kdy schopnost šifrovat
$\Leftrightarrow$ schopnost dešifrovat
$\Leftrightarrow$ znalost klíče

\subsection{Diffie + Hellman (1976)}
\begin{itemize}
\item můžou být dva klíče
\item 1. popis kryptografie s veřejným klíčem (před tím vynalezeno britskou tajnou službou)
\item public key slouží pro šifrování
\item kdokoliv může šifrovat
\item veřejná databáze veřejných klíčů
\item soukromý klíč umožňuje dešifrování
\item soukromý klíč nesmí být možné odvodit od veřejného klíče
\end{itemize}

Předpoklad existence databáze veřejných klíčů je netriviální
-- dosud neexistuje důvěryhodné řešení $\rightarrow$ samostatná přednáška o spravování (někde v přednáškách o aplikované kryptografii)


\subsection{Definice Public key encryption scheme}
\begin{itemize}
\item trojice algoritmů $(G, E, D)$:
\item $G$ - generuje dvojice klíčů $(pK, sK) \leftarrow G(^)$
\item $E$ - pravděpodobnostní algoritmus $c \leftarrow E_{pK}(m)$
\item $D$ - deterministický algoritmus $m = D_{sK}(c)$
\end{itemize}

Korektnost:
    $\forall (m, (pK, sK) \leftarrow G(1^n)) D_{sK}(E_{pK}(m)) = m$

\subsection{Definice Indistinguishability ciphertextů}

Nechť $(G,E,D)$ je PKE na prostoru všech správ $\mathcal{M} = \bigcup_{i \in \mathbb{N}}\mathcal{M}_i$,
$\forall PPT A$ existuje negligible $\epsilon()$,
$$\forall m_0,m_1 \in \mathcal{M}_n |\Pr[A(1^n,pK, E_{pK}(m_0))=1]-Pr[A(1^n,pK,E_{pK}(m_1))=1]| \le \epsilon(n) \forall n$$
kde pravděpodobnost je přes $(pK,sK) \leftarrow G(1^n)$ a náhodné mince $A$ a $E$.


$\Rightarrow$ pro PKE máme ekvivalenci ind. ciphertextů a semantic security



Pro PKE potřebujeme kvalitativně silnější předpoklady
    nestačí jen existence jednosměrných funkcí
    potřebujeme jednosměrné permutace se zadními vrátky (trapdoor permutace)
        one way permutation + trapdoor


\subsection{Definice Kolekce trapdoor permutací}
Kolekce funkcí $F = {f_k \times D_k \rightarrow D_k}$ je kolekcí trapdoor permutací $\iff$
\begin{itemize}
\item pro všechny $k$ je $f_k$ permutací
\item existuje generátor $(k,t)$ kde $t$ je trapdoor
\item znalost $k$ umožňuje efektivně vybrat efektivně z rovnoměrného rozdělení na $D_k$
\item znalost $k$ umožňuje efektivně vyhodnotit $f_k(x)$ pro všechny $x \in D_k$
\item $\forall PPT A \exists negl. \epsilon() \Pr[A(1^n,K,f_K(X)) \in f_K^{-1}(f_K(n))] \le \epsilon(n) \forall n$
    pro $(K,T) \leftarrow G(1^n), X \leftarrow D_k$ a pro náhodné mince $A$

\item znalost $T$ umožňuje efektivně nalézt $f_k^{-1}(y)$ pro všechna $(k,t) \leftarrow G(1^n)$
\end{itemize}

Pro PKE potřebujeme schopnost vygenerovat "těžký problém a jeho řešení"
(nemůžeme si vybrat ten problém)
Pro trapdoor funkce máme výrazně méně kandidátů než pro jednosměrné funkce
%obecně mnohem méně heuristických kandidátů než pro jednosměrné funkce

\subsection{Příklad: RSA}
\begin{itemize}
\item    kolekce $f_{N,e}: Z_{N}* \rightarrow Z_{N}*$
\item    $N = pq$
\item    $e \leftarrow Z* \phi(N)$
\item    $f_{N,e}(x) = x^e \mod N$
\item    $t = d; ed = 1 \mod \phi(N)$

\item    inverzní zobrazení:
    $f_{N,e}^{-1}(y,d) = y^d \mod N$

\item    $y = x^e \mod N => f(y,d)^{-1} = y^d \mod N = x^ed \mod N = x \mod N$

\item    $pK = (e,N)$

\item    $sK = (d)$
\end{itemize}

Jak generátor spočítá $k,t$?
\begin{itemize}
\item    $k = e,N$
\item    $d = e^{-1}$ v $Z_{\phi N}*$ (pomocí rozšířeného Eukleidova algoritmu)
\end{itemize}





\subsection{Rabinova kolekce TDPs}
\begin{itemize}
\item    parametry $p \equiv q \equiv 3 (\mod 4)$, $N = pq$
\item    trapdoor $(p,q)$
\item    $fN: QR_N \rightarrow QR_N fN(x) = x^2 mod N$
\end{itemize}

    inverze:
        pro $a$ z $QR$ hledáme $z$ takové, aby $a \equiv z^2 mod N$

        $Z_{pq}* \approx Z_p* x Z_q*$

        pro prvočísla se druhá odmocnina hledá velmi snadno

        $z_p \equiv a^{p+1 / 4} mod p$
        $z_q \equiv a^{q+1 / 4} mod p$

        stačí se podívat na $\pm \alpha z_p \pm \beta z_q$
        $\alpha \equiv 1 \mod p$
        $\alpha \equiv 0 \mod q$
        $\beta \equiv 0 \mod p$
        $\beta \equiv 1 \mod q$

        dostáváme 4 druhé odmocniny $a$ v $Z_{N}*$
        právě jedna bude v $QR_N$


Konstrukce PKE z TDPs
    1)
        vyber $(k,t) \leftarrow G(1^n)$
    2)
        $pK = k$
        $sK = t$
        $E_{pK}(x) = f_k(x)$
        $D_{sK}(y) = f_k^{-1}(y)$

INSECURE!!! - Pro RSA nefunguje například pro malé $e$

těch problémů je tam víc
    všechny RSA standardy mluví o paddingu
    je potřeba ty zprávy nějak náhodně rozdistribuovat




$OWF \not\rightarrow OWP$

$OWF \not\rightarrow PKE$

Otevřený problém:
    konstrukce PKE z OWF (když ale nebudu OWF používat jako blackbox)?

Tím, že máme málo kandidátů, hrozí jejich vyřešení (mimo jiné kvantovým počítačem)




konstrukce z hardcore bitů

$M = {0,1}$

$(k,t) \leftarrow G(1^n)$

$pK = k$

$sK = t$

$E_{pK}(m) : m \leftarrow Dk$
    vrať $fk(x), bk(x) \oplus m$

$D_{sK}(c) : c = c_1, c_2$
$f_k^{-1}(c_1) = \tilde{x}$
    vrať $b_k(X) \oplus c_2$


Tvrzení:
    pokud $\{f_k\}$ je kolekce TDPs s hardcore bity $\{b_k\}$
    pak předešlé schéma splňuje ind. ciphertextů

    adversary nemůže rozpoznat $b_k$ od náhodných

Myšlenka důkazu:
$$(pK, E_{pK}(0)) \equiv (k, (f_k(x), b_k(x)))
\equiv_c (k, (f_k(x), R)) \equiv (k, (f_k(x), !R))
\equiv_c (k, (f_k(x), !b_k(x))) \equiv (pK, E_{pK}(1))$$



Tohle způsobuje dvojnásobnou délku ciphertextu
Efektivnější varianta:
    $M = {0,1}^l$
    $E_{pK}(m)$:
        $x \leftarrow Dk$
        $r = G_l(x) = b_k(x)||b_k(f_k(x))||b_k(f_k^2(x))||...||b_k(f_k^{l-1}(x))$
        vrať $(f^l_k(x), r \oplus m)$
Důkaz jako pro $PRG 1 \rightarrow PRG l$
%když tak to umí Verča


-> Blum + Goldwasser
staví na Rabinově kolekci TDPs
$p \equiv q \equiv 3 \mod 4$
$N = pq$
$pK = N$
$sK = (p,q)$

$E_{pK}(m)$:
\begin{enumerate}
\item $x \leftarrow QR_N$
\item $r = lsb(x) || lsb(x^2 \mod N) || lsb(x^4 \mod N) || ... || lsb(x^{2^(l-1)} \mod N)$ -- Blum Blum Shub generátor
\item vrať $c = (x^{2^l} \mod N, r \oplus m)$
\end{enumerate}


$$x_p \equiv y^{((p+1)/4)^l} \mod p$$
$$x_q \equiv y^{((q+1)/4)^l} \mod q$$
$$X \equiv q(q^{-1} \mod  p)x_p + p(p^{-1} \mod q)x_q (\mod N)$$


\subsection{Diffie-Hellman key exchange}
mějme cyklickou grupu $G$ řádu $q$ a generátor $g$
\begin{description}
\item [Alice:] \ 
\begin{itemize}
\item    zvolí náhodně $G,q,g$
\item    vybere $x \leftarrow Z_q$
\item    $h_A = g^x$
\item    pošle $(G, q, g, h_A)$ Bobovi
\end{itemize}
\item [Bob:] \ 
\begin{itemize}
\item    vybere $y \leftarrow Z_q$
\item    $h_B = g^y$
\item    pošle $h_B$ Alici
\end{itemize}

\item [Oba:]
    $$k_A = h_B^x = g^{xy} = h_A^y = k_B$$
\end{description}

vyžaduje to aby diskrétní logaritmus byl těžký

Computational Diffie-Hellman assumption:
$$G,q,q,g^x,g^y \not\rightarrow g^xy$$

Decision Diffie-Hellman assumption:
($g^z$ = náhodný prvek $G$)

$$G,q,g,g^x,g^y,g^xy \equiv_c G,q,g,g^x,g^y,g^z$$



$\rightarrow$ El Gamal (příště (za dva týdny))

\end{document}
