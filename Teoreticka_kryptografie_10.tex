\documentclass[a4paper,12pt,titlepage]{article}
\usepackage[IL2]{fontenc}
\usepackage[utf8]{inputenc}
\usepackage{a4wide}
\usepackage[czech]{babel}
\usepackage{amsfonts, amsmath, amsthm, amssymb}
\usepackage[left=1.5cm,right=1.5cm,top=1.5cm,bottom=1.5cm]{geometry}
\usepackage{mdwlist}
\usepackage{textcomp}

\def\nadpis#1{{\bigskip\large\bf\noindent{#1}}\par\bigskip}
\def\podnadpis#1{{\bigskip\bf\noindent#1\medskip\par}}
\def\definice{\medskip\noindent {\bf Definice: }}
\def\priklad{\noindent {\bf Příklad: }}
\def\tvrzeni{\medskip\noindent {\bf Tvrzení: }}
\def\lemma{\noindent {\bf Lemma: }}
\def\veta{\noindent {\bf Věta: }}
\def\dukaz{\noindent {\bf Důkaz: }}
\def\qed{{\hfill{$\square$}}}
\def\m#1{{\mathcal{#1}}}
\def\sskip{\medskip}

\begin{document}
\podnadpis{21. 12. 2017}

\podnadpis{Zero knowledge}

Alice chce dokázat Bobovi, že $x\in L$ tak, aby Bob uvěřil, ale nebyl schopen důkaz replikovat. Zero knowledge má mnoho aplikací v kryptografii.

\noindent{\it Příklady:}

Alice má číslo $N=pq$ a chce dokázat Bobovi, že $N$ je složené. Navíc Bob se z důkazu nesmí dozvědět žádnou další informaci. Tedy chce to udělat tak aby nakonec Bob neznal $p$ ani $q$.

\noindent Kde je Waldo (Wally)?
Kde je Waldo (Wally)? Úkolem je najít na velkém obrázku přeplněném postavičkami najít Wallyho (konkrétní postavička). Představte si, že chcete přesvědčit kamaráda, že na obrázku Waldo skutečně je, ale nechat jej ho najít. Uděláme to tak, že položíme obrázek na stůl a překryjeme jej dostatečně velkým papírem s okénkem, ve kterém je vidět pouze Waldo. Kamarád může v okénku najít Walda. Tím bychom jej ale nepřesvědčili -- mohli jsme podstrčit jiný obrázek. Proto okénko zakryjeme a povolíme udělat právě jednu z těchto dvou akcí
\begin{enumerate}
	\item  odkrýt okénko -- najít Walda a přesvědčit se, že na obrázku je.
	\item  odkrýt celý obrázek -- přesvědčit se, že není podstrčen jiný obrázek
	\end{enumerate}
	Kamarád se nechá přesvědčit pokud při náhodně zvolené akci dostane správnou odpověď (na obrázku je Waldo nebo na stole je obrázek).
	Pokud na vybraném obrázku Waldo není může být kamarád podveden s pravděpodobností 1/2 (musel si špatně zvolit akci). Pokud Waldo na obrázku je bude kamarád přesvědčen vždy.

\podnadpis{Interaktivní protokoly}

Alice a Bob jsou interaktivní Turingovy stroje se společným vstupem $x$ , které sdílejí komunikační pásku. Interakce probíhá v kolech.

Exekuce těchto Turingových strojů je dvojice
$$(V_A,V_B)=((x,z_1,r_1,M_a),(x,z_2,r_2,M_B)),$$
kde $x$ je společný vstup, $z_1$ (resp. $z_2$) případný soukromý vstup Alice (Boba), $r_1,r_2$ náhodné řetězce a $M_A$ zprávy od Boba (podobně $M_B$ zprávy od Alice). Tedy
$$M_A=\{m_A^1,m_A^2, \dots\},\ M_B=\{m_B^1,m_B^2, \dots\}$$
$$m_A^i,m_B^i,\ x,r_i,z_i\in \{0,1\}^*.$$

Interakci mezi Turingovými stroji $(A,B)$ je náhodná
proměnná, kterou značíme $A_{r_1}(x,z_1)\leftrightarrow
B_{r_2}(x,z_2)$, kde $r_1,r_2 \leftarrow \{0,1\}^k$.

$V_A = \left(x,z_1,r_1, M_A  \right)$ je view $A$, podobně $V_B = \left(x,z_2,r_2, M_B\right)$ je view $B$.
Dále $out_x(e)$, kde $\ x\in\{A,B\}$, značí výstup stroje $x$ a $m^i_A$ je $i$-tá zpráva od Boba.

\podnadpis{Interactive proof}

Dvojice interaktivních Turingových strojů $P$ a $V$ je interaktivní důkaz pro jazyk $L$, pokud $V$ je PPT a platí:

\begin{enumerate}

\item {\bf Completeness:} $\forall x\in L\ \exists y\in\{0,1\}^*$ t.ž. $\forall z\in\{0,1\}^*:$
$$\Pr[{\rm out}_V[P(x,y)\leftrightarrow V(x,z)]=1]=1\ 
%\footnote{Tedy pro všechny $x$ z jazyka $L$ prover dokáže verifiera přesvědčit.} 
\footnote{Poznamenejme, že můžeme požadovat pouze, že pravděpodobnost přijetí je alespoň $c$ .}$$
\item {\bf Soundness:} $\exists$ negligible $\varepsilon$ t.ž. $\forall x\notin L,\ \forall \text{ Turingovy stroje } P^*$ a $\forall z\in\{0,1\}^*$:
	$$\Pr[{\rm out}_V[P^*(x)\leftrightarrow V(x,z)]=1]\leq\varepsilon(n),\footnote{
	$P^*$ nemá vstup $y$, protože prover ji v sobě může mít zadrátovánu.} \footnote{Poznamenejme, že můžeme požadovat, že pravděpodobnost přijetí je nejvýše nějaké $s$}$$
	kde $|x|=n$.

\end{enumerate}

Definujeme IP jako třídu jazyků, které mají interaktivní důkaz.

Pozorování: NP$\subseteq$IP.

Věta (Shamir): IP=PSPACE

\podnadpis{Interaktivní důkaz pro neizomorfismus grafů}

Vstupem jsou dva grafy $G_0=(V_0,E_0), G_1=(V_1,E_1)$ (oba
na $n$ vrcholech), chceme rozhodnout, jestli jsou izomorfní (tj. $\exists \sigma\in
S_n$ t.ž. $(u,v) \in E_0  \leftrightarrow (\sigma(u), \sigma(v)) \in E_1$ a značíme $\sigma(G_0)=G_1$).

Izomorfismus grafů budeme značit $L_{\rm ISO}\in\rm NP$ a neizomorfismus grafů $L_{\rm NISO}\in\rm coNP$.

\podnadpis{Protokol pro $L_{\rm NISO}$}

Společný vstup je $X=(G_0,G_1)$

\begin{enumerate}

\item Verifier $V(X)$ zvolí náhodně bit $b$ a permutaci $\sigma$.

\item Pošle proverovi $H=\sigma(G_b)$.

\item Prover najde $b'$ t.ž. $H\sim G_{b'}$.

\item Pošle $b'$ verifierovi.

\item Verifier vrátí 1, pokud $b=b'$

\end{enumerate}

Celý postup opakuj $n$-krát.

\tvrzeni $(P,V)$ je interaktivní důkaz pro $L_{\rm NISO}$.

\dukaz Completeness: $x\in L_{\rm NISO} \Rightarrow$ Prover $P$ spočítá, který z grafů mu byl poslán (poznamenejme, že $P$ je unbounded) a tedy vždy určí $b'=b$.

\noindent Soundness: $x\in L_{\rm ISO} \Rightarrow$ Prover není schopen určit jestli dostal $G_0$ nebo $G_1$ a tedy pravděpodobnost úspěchu při jednom opakování je právě $\frac{1}{2}$. Při $n$ opakování dostaneme pravděpodobnost úspěchu nejvýš $\frac{1}{2}^n$. \qed

Zero cash, měny s anonymitou, existují prakticky použitelné implementace pro kryptografické problémy a pro přirozené jazyky v NP. 

\podnadpis{Efektivní interaktivní důkaz pro grafový isomorfismus}

Společný vstup $X=(G_0,G_1)$ takové, že $|V_0|=|V_1|=n$, svědkem pro $P$ je $\sigma$ t.ž. $\sigma(G_1)=G_0$.

\begin{enumerate}

	\item Prover zvolí $\pi\leftarrow S_n$ a pošle
		verifierovi graf $H=\pi(G_0)$.

	\item  Verifier zvolí bit $b$ a pošle jej proverovi

	\item  Pokud $b=0$, prover pošle $\pi$, jinak pošle $\pi'=\pi\circ\sigma$ (tedy $H=\pi'(G_1)$)

	\item  Verifier ověří, že dostal správnou permutaci.

\end{enumerate}

Opakujeme $n$-krát.


\tvrzeni $(P,V)$ je interaktivní důkaz pro $L_{\rm ISO}$.

\dukaz\ Completeness: Jsou-li grafy izomorfní, prover umí pro oba případy odpovědět.

Soundness: Nejsou-li izomorfní, v jednom pokusu umí prover odpovědět nejvýše s pravděpodobností $1/2$, neboť $H$ může být isomorfní jen jednomu grafu. 
\qed

Poznamenejme, že verifier se nedozvěděl $\sigma$. Tedy protokol bude i zero-knowledge. Viz důkaz dále.

\definice (Honest verifier zero knowledge)

Nechť $P_V$ je interaktivní důkaz pro jazyk $L\in NP$ se svědeckou relací $R_L$.

Řekneme, že $(P,V)$ je \emph{Honest verifier zero knowledge}, pokud existuje
PPT simulátor takový, že pro každého PPT distinguishera $D$ existuje negligible
$\varepsilon$ takové, že $\forall x\in L,y\in R_L(x)$ a $z\in\{0,1\}^*$: $D$
rozliší následující distribuce s pravděpodobností nejvýše $\varepsilon(|x|)$

$$\{\text{view}_V[P(x,y)\leftrightarrow V(x,z)]\} \text{ vs. }\{S(x,z)\}.$$

\definice (Zero knowledge)

$\forall$ PPT $ V^*$ existuje expected PPT\footnote{očekávaný polynomiální čas}
simulátor $S$ takový, že pro každého PPT distinguishera $D$ existuje negligible
$\varepsilon$ takové, že $\forall x\in L,y\in R_L(x)$ a $z\in\{0,1\}^*$: $D$
rozliší následující distribuce s pravděpodobností nejvýše $\varepsilon(|x|)$


$$\{\text{view}_{V^*}[P(x,y)\leftrightarrow V^*(x,z)]\} \text{ vs. }\{S(x,z)\}.$$


\tvrzeni $(P,V)$ je zero knowledge interaktivní důkaz pro $L_{\rm ISO}$

\dukaz Potřebujeme sestrojit simulátor $S(x,z)$ pro každého verifiera $V^*$, pro jednu iteraci protokolu bude dělat následující:

\begin{enumerate}

	\item Vyber bit $b'\leftarrow \left\{ 0,1 \right\}$ a permutaci $\pi \leftarrow S_n$ nechť  $H=\pi(G_b)$.

	\item Emuluj $V_r^*$ na vstupu $x,z$ s náhodnými bity $r\leftarrow\{0,1\}^k$ se zprávou $H$ od Provera.

		Z emulace dostaneme $b=V^*_r(x, z,H)$.

	\item Pokud $b=b'$, vrať view$_{V^*}(x,z,r,H,b,\pi)$, jinak se vrať na začátek a postup opakuj.

\end{enumerate}

Potřebujeme ukázat:
a) $S$ pracuje v očekávaném polynomiálním čase.

b) Distribuce ze simulátoru $S$ a view verifiera $V^*$ jsou nerozlišitelné. 


Add b)
Nejprve dokážeme pomocné lemma:

\lemma Při exekuci $S(x,z)$ má $H$ stejnou distribuci jako $\pi(G_0)$ a $\Pr[b'=b]=\frac{1}{2}$

Pokud $G_0, G_1$ jsou izomorfní, pak $\{\pi(G_0\}$ a $\{\pi(G_1)\}$ jsou stejné distribuce $\Rightarrow$ Distribuce $H,\pi(G_0)$ jsou stejné.

	Distribuce $H,b'$ jsou nezávislé (protože $G_0,G_1$ jsou isomorfní) $\Rightarrow$ $V^*$ má na vstupu pouze $H$, to nezávisí na $b'$ a navíc $b'$ je rovnoměrně rozdělené, tedy $\Pr[b'=b]=\frac{1}{2}$. Tedy distribuce $H$ nezávisí na $b'$, stejně tak distribuce $\pi$ nezávisí na $b'$ a tedy jsou distribuce $H$ a $\pi(G_0)$ stejné.

Add a)
Protože očekávaný počet iterací algoritmu, než vrátí view jsou 2 (v každé iteraci má pravděpodobnost $\frac{1}{2}$ na skončení dle lemmatu) a jednotlivé iterace jsou v polynomiálním čase.


\end{document}
