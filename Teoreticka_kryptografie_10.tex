\documentclass[a4paper,12pt,titlepage]{article}
\usepackage[IL2]{fontenc}
\usepackage[utf8]{inputenc}
\usepackage{a4wide}
\usepackage[czech]{babel}
\usepackage{amsfonts, amsmath, amsthm, amssymb}
\usepackage[left=1.5cm,right=1.5cm,top=1.5cm,bottom=1.5cm]{geometry}
\usepackage{mdwlist}
\usepackage{textcomp}

\def\nadpis#1{{\bigskip\large\bf\noindent{#1}}\par\bigskip}
\def\podnadpis#1{{\bigskip\bf\noindent#1\medskip\par}}
\def\definice{\medskip\noindent {\bf Definice: }}
\def\priklad{\noindent {\bf Příklad: }}
\def\tvrzeni{\medskip\noindent {\bf Tvrzení: }}
\def\veta{\noindent {\bf Věta: }}
\def\dukaz{\noindent {\bf Důkaz: }}
\def\qed{{\hfill{$\square$}}}
\def\m#1{{\mathcal{#1}}}
\def\sskip{\medskip}

\begin{document}
\podnadpis{21. 12. 2017}

\podnadpis{Zero knowledge}

Co když Alice nevěří Bobovi?

\noindent Alice chce dokázat Bobovi, že $x\in L$ tak, aby Bob uvěřil, ale nebyl schopen důkaz replikovat.

\noindent Alice má $N=pq$ a chce dokázat, že je složené, tak, aby Bob nevěděl $p$ a $q$ (crypto aplikace)

\noindent Kde je Waldo (Wally)?

Položíme knížku na stůl a překryjeme dostatečně velkým papírem s okénkem. Buď v okénku najdeme Wallyho, nebo pod papírem knížku, ale nemáme informaci o tom, kde se Wally na stránce nachází.

\podnadpis{Interaktivní protokoly}

Alice, Bob, interaktivní TS

mají společný vstup $x$ a sdílejí komunikační pásky

interakce probíhá po kolech

Exekuce je dvojice
$$(V_A,V_B)=((x,z_1,r_1,M_a),(x,z_2,r_2,M_B)),$$
kde $x$ je vstup, $z$ anxiliary input, $r$ mince a $M_A$ zprávy od Boba (podobně $M_B$). Platí:
$$M_a=\{m_A^1,m_A^2, \dots\},\ M_B=\{m_B^1,m_B^2, \dots\},\ x,r_i,z_i\in \{0,1\}^*.$$

Interakci mezi TM$(A,B)$ značíme $A(r_1,z_1)\leftrightarrow B(r_2,z_2)$ a mějme odpovídající náhodné proměnné (pokud $r_1,r_2\in \{0,1\}^k$): $V_A$ je view $A$, podobně $V_B$ je view $B$, out$_x(e),\ x\in\{A,B\}$ výstup příslušného stroje a $m^i_A$ je $i$tá zpráva od Boba

\podnadpis{Interactive proof}

$P$ prover, $V$ verifier

Dvojice interaktivních TS $A$ a $B$ je interaktivní důkaz pro jazyk $L$, pokud $V$ je PPT a platí:

\begin{enumerate}

\item {\bf Completeness} \dots $x\in L\ \exists\, y\in\{0,1\}^*$ t.ž. $\forall z\in\{0,1\}^*:$
$${\rm Pr}[{\rm out}_V[P(x,y)\leftrightarrow V(x,z)]=1]=1$$
\item {\bf Soundness} \dots $\exists\, {\rm reg}!\, z$ t.ž. $\forall x\notin L,\ \forall {\rm PPT\ } P^*$ a $\forall z\in\{0,1\}^*$ ($|x|=n$):
$${\rm Pr}[{\rm out}_V[P^*(x)\leftrightarrow V(x,z)]=1]\leq\epsilon(n)$$(Tady může mít prover informaci $y$ v sobě zakódovanou)

\end{enumerate}

Buď IP třída jazyků, které mají interaktivní důkaz (NP$\,\subseteq\,$IP)

Shamir: IP=PSPACE

\podnadpis{Interaktivní důkaz pro neizomorfismus grafů}

$G_0=(V_0,E_0), G_1=(V_1,E_1)$ (oba na $n$ vrcholech), definuji izomorfismus $\exists \sigma\in S_n$ t.ž. $\sigma(G_0)=G_1$.

Izomorfismus grafů $L_{\rm ISO}\subset\rm NP$, neizomorfismus grafů $L_{\rm NISO}\subset\rm coNP$.

\podnadpis{Protokol pro $L_{\rm NISO}$}

Společný vstup je $X=(G_0,G_1)$

\begin{enumerate}

\item Verifier $V(X)$ zvolí náhodně bit $b$ a permutaci $\sigma$.

\item Pošle proverovi $H=\sigma(G_b)$.

\item Prover najde $b'$ t.ž. $H\sim G_{b'}$.

\item Pošle $b'$ verifierovi.

\item Verifier vrátí 1, pokud $b=b'$

\end{enumerate}

Opakuj $n$krát.

\tvrzeni $(P,V)$ je interaktivní důkaz pro $L_{\rm NISO}$.

\dukaz Completeness: $x\in L_{\rm NISO} \Rightarrow P$ vždy najde $b'=b$

\noindent Soundness: $x\in L_{\rm ISO} \Rightarrow$ pst. úspěchu v jednom pokusu je právě $1/2$ \qed
\medskip

\noindent{\bf Najít si:} zero cash, měny s anonymitou, prakticky použitelné aplikace pro kryptografické problémy
\smallskip

Pro přirozené jazyky v NP program dává nějakou hodnotu na vstupu%??????????????????

\podnadpis{Exektivní interaktivní důkaz pro G.ISOMORFISMUS}

Společný vstup $X=(G_0,G_1)$, $|V_0|=|V_1|$

svědkem pro $P$ je $\sigma$ t.ž. $\sigma(G_1)=G_0$.

Prover zvolí $\pi\leftarrow S_n$, $H=\pi(G_0)$ a pošle $V$ graf $H$.

$V$ zvolí bit $b$ a pošle proverovi

Pokud $b=0$, $P$ pošle $\pi$, jinak pošle $\pi'=\pi\circ\sigma$ (tedy $H=\pi'(G_1)$)

Verifier ověří, že dostal správný izomorfismus.

Opakujeme $n$krát.

\tvrzeni $(P,V)$ je interaktivní důkaz pro $L_{\rm ISO}$.

\dukaz\ Completeness: Jsou-li grafy izomorfní, prover umí pro oba případy odpovědět.

Soundness: Nejsou-li izomorfní, v jednom pokusu umí prover odpovědět nejvýš s pravděpodobností $1/2$, neboť $H$ může být izomorfní jen jednomu grafu. 

Zero knowledge: V se z protokolu nedozví vůbec nic.\qed

\definice (Honest verifier zero knowledge)

Nechť $P_V$ je interaktivní důkaz pro jazyk $L\in NP$ se svědeckou relací $R_L$. Bere svědecké relace a vrací 0 nebo 1.

Řekneme, že $(P,V)$ je Honest verifier zero knowledge, pokud existuje PPT simulátor t.ž. $\forall$ PPT distinguishery $D$ $\exists$ negligible $\varepsilon:\forall x\in L,y\in R_L(x)$ a $z\in\{0,1\}^*$ $D$ rozliší následující distribuce s pravděpodobností nejvýše $\varepsilon(|x|)$:

$\{$view$_V[P(x,y)\leftrightarrow V(x,z)]\}$ vs. $\{S(x,z)\}$

\definice (Zero knowledge)

	$\forall$PPT$\ P^*$ $\exists$ expected PPT (očekávaný polynomiální čas) simulátor $$S:\ \{{\rm view}_{V^*}[P(x,y)\leftrightarrow V^*(x,y)]\}$$

\tvrzeni $(P,V)$ je zero knowledge důkaz pro $L_{\rm ISO}$

\dukaz simulátor $S(x,z)\ \forall V^*$ pro jednu iteraci

\begin{enumerate}

\item vyber bit $b'$, $\pi {\rm \ in\ } S_n$:\ $H=\pi(G_b)$

\item emuluj $V_r^*$ se vstupem $x,z$ pro $r\leftarrow\{0,1\}^k$ na vstupu $H$ pro $k$

dostaneme $b=V^*_r(x, z,H)$

\item pokud $b=b'$, vrať view$_{V^*}(x,z,r,H,b,\pi)$, jinak se vrať do 2. a opakuj

\end{enumerate}

a) S pracuje v očekávaném poly čase, distribuce jsou stejné

b) při exekuci $S(x,z)$ má $H$ stejnou distribuci jako $\pi(G_0)$ a Pr$[b'=b]=\frac{1}{2}$

když $g_0, G_1$ jsou izomorfní, pak $\{\pi(G_0\}$ a $\{\pi(G_1)\}$ jsou stejné $\Rightarrow$ distribuce $H,\pi(G_0)$ jsou stejné.

distribuce $H,b'$ jsou nezávislé $\Rightarrow$ $V^*$ má na vstupu pouze $H$, které nezávisí na $b'$, protože $b'$ je rovnoměrně rozdělené

a) z lemmatu, protože očekávaný počet iterací jsou 2 a jednotlivé iterace jsou PPT

b) $H$ nezávisí na $b'$, $\pi$ nezávisí na $b'$ $\Rightarrow$ distribuce $\pi$ se nezmění

\end{document}